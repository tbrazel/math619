\usepackage{amsmath,amssymb,amsthm}
\usepackage{mathrsfs}
\usepackage{bm,bbm}
\usepackage{color}
\usepackage[rgb,dvipsnames]{xcolor}
\usepackage{mathtools}

\usepackage{url}

\usepackage{epigraph}
\setlength{\epigraphwidth}{0.8\textwidth}

\usepackage{tikz}
\usetikzlibrary{shapes.geometric, arrows}
\usetikzlibrary{decorations.pathmorphing}
\usetikzlibrary{cd}
\tikzstyle{var} = [rectangle, minimum width=1cm, minimum height=0.5cm, text centered, draw=black, fill=white!30]
\tikzstyle{output} = [rectangle, text centered, draw=black, fill=gray!20]
\tikzstyle{input} = [rectangle, text centered, draw=black, fill=green!20]
\tikzstyle{arrow} = [thick,->,>=stealth]
\tikzset{
  closed/.style = {decoration = {markings, mark = at position 0.5 with { \node[transform shape, xscale = .8, yscale=.4] {/}; } }, postaction = {decorate} },
  open/.style = {decoration = {markings, mark = at position 0.5 with { \node[transform shape, scale = .7] {$\circ$}; } }, postaction = {decorate} }
}

\setlength{\parindent}{0pt} 
  
  
\let\fullref\autoref
%
%  \autoref is very crude.  It uses counters to distinguish environments
%  so that if say {lemma} uses the {theorem} counter, then autrorefs
%  which should come out Lemma X.Y in fact come out Theorem X.Y.  To
%  correct this give each its own counter eg:
%                 \newtheorem{theorem}{Theorem}[section]
%                 \newtheorem{lemma}{Lemma}[section]
%  and then equate the counters by commands like:
%                 \makeatletter
%                   \let\c@lemma\c@theorem
%                  \makeatother
%
%  To work correctly the environment name must have a corrresponding 
%  \XXXautorefname defined.  The following command does the job:
%
\def\makeautorefname#1#2{\expandafter\def\csname#1autorefname\endcsname{#2}}
%
%  Some standard autorefnames.  If the environment name for an autoref 
%  you need is not listed below, add a similar line to your TeX file:
%  
\makeautorefname{equation}{Equation}%
\makeautorefname{footnote}{footnote}%
\makeautorefname{item}{item}%
\makeautorefname{figure}{Figure}%
\makeautorefname{table}{Table}%
\makeautorefname{part}{Part}%
\makeautorefname{appendix}{Appendix}%
\makeautorefname{chapter}{Chapter}%
\makeautorefname{section}{Section}%
\makeautorefname{subsection}{Section}%
\makeautorefname{subsubsection}{Section}%
\makeautorefname{paragraph}{Paragraph}%
\makeautorefname{subparagraph}{Paragraph}%
\makeautorefname{theorem}{Theorem}%
\makeautorefname{thm}{Theorem}%
\makeautorefname{addm}{Addendum}%
\makeautorefname{mainthm}{Main theorem}%
\makeautorefname{corollary}{Corollary}%
\makeautorefname{cor}{Corollary}%
\makeautorefname{lemma}{Lemma}%
\makeautorefname{lem}{Lemma}%
\makeautorefname{sublemma}{Sublemma}%
\makeautorefname{sublem}{Sublemma}%
\makeautorefname{subl}{Sublemma}%
\makeautorefname{prop}{Proposition}%
\makeautorefname{property}{Property}
\makeautorefname{pro}{Property}
\makeautorefname{sch}{Scholium}%
\makeautorefname{step}{Step}%
\makeautorefname{conject}{Conjecture}%
\makeautorefname{conjecture}{Conjecture}%
\makeautorefname{questn}{Question}
\makeautorefname{quest}{Question}
\makeautorefname{qn}{Question}
\makeautorefname{definition}{Definition}%
\makeautorefname{defn}{Definition}%
\makeautorefname{defi}{Definition}%
\makeautorefname{def}{Definition}%
\makeautorefname{dfn}{Definition}%
\makeautorefname{df}{Definition}%
\makeautorefname{notation}{Notation}
\makeautorefname{notn}{Notation}
\makeautorefname{rem}{Remark}%
\makeautorefname{remark}{Remark}%
\makeautorefname{rems}{Remarks}%
\makeautorefname{rmk}{Remark}%
\makeautorefname{rk}{Remark}%
\makeautorefname{remarks}{Remarks}%
\makeautorefname{rems}{Remarks}%
\makeautorefname{rmks}{Remarks}%
\makeautorefname{rks}{Remarks}%
\makeautorefname{example}{Example}%
\makeautorefname{examp}{Example}%
\makeautorefname{exmp}{Example}%
\makeautorefname{exam}{Example}%
\makeautorefname{exa}{Example}%
\makeautorefname{axiom}{Axiom}%
\makeautorefname{axi}{Axiom}%
\makeautorefname{ax}{Axiom}%
\makeautorefname{case}{Case}%
\makeautorefname{claim}{Claim}%
\makeautorefname{clm}{Claim}%
\makeautorefname{assumpt}{Assumption}%
\makeautorefname{asses}{Assumptions}%
\makeautorefname{conclusion}{Conclusion}%
\makeautorefname{concl}{Conclusion}%
\makeautorefname{conc}{Conclusion}%
\makeautorefname{cond}{Condition}%
\makeautorefname{const}{Construction}%
\makeautorefname{con}{Construction}%
\makeautorefname{criterion}{Criterion}%
\makeautorefname{criter}{Criterion}%
\makeautorefname{crit}{Criterion}%
\makeautorefname{exercise}{Exercise}%
\makeautorefname{exer}{Exercise}%
\makeautorefname{exe}{Exercise}%
\makeautorefname{problem}{Problem}%
\makeautorefname{problm}{Problem}%
\makeautorefname{prob}{Problem}%
\makeautorefname{prob}{Problem}%
\makeautorefname{soln}{Solution}%
\makeautorefname{sol}{Solution}%
\makeautorefname{sum}{Summary}%
\makeautorefname{oper}{Operation}%
\makeautorefname{obs}{Observation}%
\makeautorefname{ob}{Observation}%
\makeautorefname{conv}{Convention}%
\makeautorefname{cvn}{Convention}%
\makeautorefname{warn}{Warning}%
\makeautorefname{note}{Note}%
\makeautorefname{fact}{Fact}%
\makeautorefname{ouch0}{Counterexample}%
%
%                  *** End of hyperref stuff ***



\theoremstyle{definition}
\newtheorem{thm}{Theorem}[section]
\newtheorem{theorem}{Theorem}[section]

\newtheorem{addm}[theorem]{Addendum}
\newtheorem{assumptions}[theorem]{Assumptions}
\newtheorem{claim}[theorem]{Claim}
\newtheorem{conjecture}[theorem]{Conjecture}
\newtheorem{corollary}[theorem]{Corollary}
\newtheorem{cor}[theorem]{Corollary}
\newtheorem{counterexample}[theorem]{Counterexample}
\newtheorem{ouch0}[theorem]{Counterexample}
\newtheorem{df}[theorem]{Definition}
\newtheorem{defn}[theorem]{Definition}
\newtheorem{definition}[theorem]{Definition}
\newtheorem{digression}[theorem]{Digression}
\newtheorem{exmp}[theorem]{Example}
\newtheorem{example}[theorem]{Example}
\newtheorem{examples}[theorem]{Examples}
\newtheorem{exer}[theorem]{Exercise}
\newtheorem{exercise}[theorem]{Exercise}
\newtheorem{fact}[theorem]{Fact}
\newtheorem{intuition}[theorem]{Intuition}
\newtheorem{lem}[theorem]{Lemma}
\newtheorem{lemma}[theorem]{Lemma}
\newtheorem{notn}[theorem]{Notation}
\newtheorem{notation}[theorem]{Notation}
\newtheorem{note}[theorem]{Note}
\newtheorem{obs}[theorem]{Observation}
\newtheorem{observation}[theorem]{Observation}
\newtheorem{prob}[theorem]{Problem}
\newtheorem{problem}[theorem]{Problem}
\newtheorem{property}[theorem]{Property}
\newtheorem{proposition}[theorem]{Proposition}
\newtheorem{prop}[theorem]{Proposition}
\newtheorem{quest}[theorem]{Question}
\newtheorem{question}[theorem]{Question}
\newtheorem{rem}[theorem]{Remark}
\newtheorem{rmk}[theorem]{Remark}
\newtheorem{remark}[theorem]{Remark}
\newtheorem{rems}[theorem]{Remarks}
\newtheorem{remarks}[theorem]{Remarks}
\newtheorem{scholium}[theorem]{Scholium}
\newtheorem{terminology}[theorem]{Terminology}
\newtheorem{upshot}[theorem]{Upshot}
\newtheorem{warn}[theorem]{Warning}
\newtheorem{warning}[theorem]{Warning}

\newtheorem*{mainthm}{Main Theorem}
\newtheorem{introthm}{Theorem}
\newtheorem*{introcor}{Corollary}
\newtheorem*{introconj}{Conjecture}
\renewcommand*{\theintrothm}{\Alph{introthm}}

%%%% hack to get fullref working correctly
\makeatletter
\let\c@cor=\c@thm
\let\c@prop=\c@thm
\let\c@lem=\c@thm
\let\c@conjecture=\c@thm
\let\c@defn=\c@thm
\let\c@df=\c@thm
\let\c@exmp=\c@thm
\let\c@rem=\c@thm
\let\c@sch=\c@thm
\let\c@con=\c@thm
\let\c@notn=\c@thm
\let\c@example=\c@thm
\let\c@remark=\c@thm
\let\c@equation\c@thm
\makeatother

  


\newenvironment{proofsketch}{%
 \renewcommand{\proofname}{{Proof Sketch}\proof}{\endproof}}

\newcommand{\ab}{\text{ab}}
\newcommand{\Aut}{\text{Aut}}
\newcommand{\BGL}{\text{BGL}}
\newcommand{\BO}{\text{BO}}
\newcommand{\BU}{\text{BU}}
\renewcommand{\char}{\text{char}}
\newcommand{\cl}{\text{cl}}
\newcommand{\coeq}{\text{coeq}}
\newcommand{\cof}{\text{cof}}
\DeclareMathOperator*{\mycolim}{colim}
\newcommand{\colim}{\mathop{\mycolim}}
\newcommand{\conj}{\text{conj}}
\newcommand{\const}{\text{const}}
\newcommand{\DM}{\text{DM}}
\newcommand{\End}{\text{End}}
\newcommand{\ev}{\text{ev}}
\newcommand{\Ext}{\text{Ext}}
\newcommand{\Frac}{\text{Frac}}
\newcommand{\Frob}{\text{Frob}}
\newcommand{\Fun}{\text{Fun}}
\newcommand{\GL}{\text{GL}}
\newcommand{\Gr}{\text{Gr}}
\newcommand{\GW}{\text{GW}}
\newcommand{\Ho}{\text{Ho}}
\newcommand{\hocof}{\text{hocof}}
\DeclareMathOperator*{\myhocolim}{hocolim}
\newcommand{\hocolim}{\mathop{\myhocolim}}
\newcommand{\hofib}{\text{hofib}}
\DeclareMathOperator*{\myholim}{holim}
\newcommand{\holim}{\mathop{\myholim}}
\newcommand{\Hom}{\textup{Hom}}
\newcommand{\hyp}{\text{hyp}}
\newcommand{\id}{\text{id}}
\newcommand{\incl}{\text{incl}}
\newcommand{\Ind}{\text{Ind}}
\newcommand{\Inn}{\text{Inn}}
\newcommand{\im}{\text{im}}
\newcommand{\KO}{\text{KO}}
\newcommand{\KU}{\text{KU}}
\newcommand{\Map}{\textup{Map}}
\newcommand{\MGL}{\text{MGL}}
\newcommand{\MO}{\text{MO}}
\newcommand{\mor}{\text{mor}}
\newcommand{\MSO}{\text{MSO}}
\newcommand{\MU}{\text{MU}}
\newcommand{\Nis}{\text{Nis}}
\newcommand{\ob}{\text{ob}}
\newcommand{\obj}{\text{obj}}
\newcommand{\op}{\text{op}}
\newcommand{\Orb}{\text{Orb}}
\newcommand{\Out}{\text{Out}}
\newcommand{\Perm}{\text{Perm}}
\newcommand{\pr}{\text{pr}}
\newcommand{\pre}{\text{pre}}
\newcommand{\Proj}{\text{Proj}}
\newcommand{\PSh}{\textit{PSh}}
\newcommand{\PShv}{\textit{PShv}}
\newcommand{\Rep}{\text{Rep}}
\newcommand{\Res}{\text{Res}}
\newcommand{\s}{\text{s}}
\newcommand{\Sch}{\text{Sch}}
\newcommand{\SH}{\mathcal{SH}}
\newcommand{\Sh}{\textit{Sh}}
\newcommand{\Shv}{\textit{Shv}}
\newcommand{\Sing}{\text{Sing}}
\newcommand{\Sm}{\text{Sm}}
\newcommand{\spn}{\text{span}}
\newcommand{\Spc}{\textit{Spc}}
\newcommand{\Spec}{\text{Spec}}
\newcommand{\SO}{\text{SO}}
\newcommand{\SP}{\text{Sp}}
\newcommand{\Stab}{\text{Stab}}
\newcommand{\SU}{\text{SU}}
\newcommand{\supp}{\text{supp}}
\newcommand{\Sym}{\text{Sym}}
\newcommand{\Th}{\text{Th}}
\newcommand{\Wr}{\text{Wr}}
\newcommand{\Zar}{\text{Zar}}
\newcommand{\sAb}{\text{sAb}}


% Categories
\newcommand{\1}{\mathbbm{1}}
\newcommand{\2}{\mathbbm{2}}
\newcommand{\Ab}{\texttt{Ab}}
\newcommand{\Cat}{\texttt{Cat}}
\newcommand{\CG}{\texttt{CG}}
\newcommand{\Ch}{\texttt{Ch}}
\newcommand{\CGWH}{\texttt{CGWH}}
\providecommand{\CRing}{\texttt{CRing}}
\newcommand{\CW}{\texttt{CW}}
\newcommand{\DDelta}{\bm{\Delta}}
\newcommand{\FinSet}{\texttt{FinSet}}
\newcommand{\Grp}{\texttt{Grp}}
\newcommand{\Grpd}{\texttt{Grpd}}
\newcommand{\Kan}{\texttt{Kan}}
\newcommand{\Poset}{\texttt{Poset}}
\providecommand{\Ring}{\texttt{Ring}}
\newcommand{\Set}{\texttt{Set}}
\newcommand{\Sp}{\texttt{Sp}}
\newcommand{\Spectra}{\texttt{Spectra}}
\newcommand{\sSet}{\texttt{sSet}}
\newcommand{\Top}{\texttt{Top}}
\newcommand{\Ring}{\texttt{Ring}}
\newcommand{\Vect}{\texttt{Vect}}


\newcommand{\CP}{\mathbb{C}\text{P}}
\newcommand{\HP}{\mathbb{H}\text{P}}
\newcommand{\RP}{\mathbb{R}\text{P}}
\newcommand{\C}{\mathscr{C}}
\newcommand{\D}{\mathscr{D}}
\newcommand{\E}{\mathscr{E}}

% Arrows
\newcommand{\po}{\arrow[ul,phantom,"\ulcorner" very near start]}
\newcommand{\pb}{\arrow[dr,phantom,"\lrcorner" very near start]}
\newcommand{\xto}[1]{\xrightarrow{#1}}
\newcommand{\from}{\leftarrow}
\newcommand{\xfrom}[1]{\overset{#1}{\leftarrow}}

\newcommand{\mapsfrom}{\mathrel{\reflectbox{\ensuremath{\mapsto}}}}
\newcommand{\longmapsfrom}{\mathrel{\reflectbox{\ensuremath{\longmapsto}}}}

\newcommand{\hookto}{\xhookrightarrow{}}
\newcommand{\xhookto}[1]{\overset{#1}{\hookrightarrow}}

\newcommand{\hookfrom}{\xhookleftarrow{}}
\newcommand{\xhookfrom}[1]{\xhookleftarrow{#1}}

\newcommand{\tto}{\twoheadrightarrow}
\newcommand{\xtto}[1]{\overset{#1}{\twoheadrightarrow}}
\newcommand{\ffrom}{\twoheadleftarrow}
\newcommand{\xffrom}[1]{\overset{#1}{\ffrom}}

\newcommand{\ladjoint}[2]{ #1\rightleftarrows #2 }
\makeatletter
\newcommand{\superimpose}[2]{%
  {\ooalign{$#1\@firstoftwo#2$\cr\hfil$#1\@secondoftwo#2$\hfil\cr}}}
\makeatother
\newcommand{\smallslash}{\mbox{\tiny/}}

\newcommand{\clhook}{\mathrel{\raisebox{0.1em}{$\mathrel{\mathpalette\superimpose{{\hspace{0.1cm}\vspace{0.1em}\smallslash}{\hookrightarrow}}}$}}}
\newcommand{\xclhook}[1]{\overset{#1}{\clhook}}

\newcommand{\clhookfrom}{\mathrel{\raisebox{0.1em}{$\mathrel{\mathpalette\superimpose{{\hspace{0.1cm}\vspace{0.1em}\smallslash}{\hookleftarrow}}}$}}}

\newcommand{\ohook}{\mathrel{\raisebox{0.03em}{$\mathrel{\mathpalette\superimpose{{\hspace{0.1cm}\vspace{0.03em}\mbox{\small$\circ$}}{\hookrightarrow}}}$}}}

\newcommand{\ohookfrom}{\mathrel{\raisebox{0.03em}{$\mathrel{\mathpalette\superimpose{{\hspace{0.1cm}\vspace{0.03em}\mbox{\small$\circ$}}{\hookleftarrow}}}$}}}

\newcommand{\cofto}{\rightarrowtail}
\newcommand{\coffrom}{\leftarrowtail}
\newcommand{\xcofto}[1]{\overset{#1}{\cofto}}
\newcommand{\xcoffrom}[1]{\overset{#1}{\coffrom}}

% Colors
\definecolor{custompurple}{RGB}{62, 34, 127}
\newcommand{\green}[1]{\textcolor{green}{#1}}
\newcommand{\red}[1]{\textcolor{red}{#1}}
\newcommand{\blue}[1]{\textcolor{blue}{#1}}
\newcommand{\purple}[1]{\textcolor{purple}{#1}}
\newcommand{\gray}[1]{\textcolor{gray}{#1} \textcolor{black}{}}

\newcommand{\ceil}[1]{\left\lceil #1 \right\rceil}
\newcommand{\floor}[1]{\left\lfloor #1 \right\rfloor}
\newcommand{\lrangle}[1]{\langle #1 \rangle}
\newcommand{\und}[1]{\underline{#1}}


\newcommand{\A}{\mathbb{A}}
%\newcommand{\C}{\mathbb{C}}
\newcommand{\F}{\mathbb{F}}
\newcommand{\N}{\mathbb{N}}
\renewcommand{\P}{\mathbb{P}}
\newcommand{\Q}{\mathbb{Q}}
\newcommand{\R}{\mathbb{R}}
\renewcommand{\S}{\mathbb{S}}
\newcommand{\U}{\mathbb{U}}
\newcommand{\V}{\mathbb{V}}
\newcommand{\W}{\mathbb{W}}
\newcommand{\Z}{\mathbb{Z}}



\newcommand{\M}{\mathcal{M}}
\renewcommand{\O}{\mathcal{O}}

\newcommand{\Cech}{\check{C}}

\renewcommand{\H}{\textbf{H}}


\let\til\widetilde
\let\oldemptyset\emptyset
\let\emptyset\varnothing
\newcommand{\es}{\emptyset}
\let\sec\S
\let\phi\varphi
\let\adj\dashv
\let\del\partial

%\renewcommand{\smash}{\wedge}
\renewcommand{\setminus}{\smallsetminus}
\newcommand{\minus}{\smallsetminus}
\renewcommand{\dot}{\bullet}
\newcommand{\iso}{\simeq}
