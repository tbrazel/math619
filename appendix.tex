% Appendix

% Thomas, 2/2 -- feel free to delete this
\section{More category theory: properties of morphisms}

As we have seen, limits and colimits allow us ways to construct new objects, morphisms, and universal properties, over indexing diagrams. In particular we might be interested in relating certain properties of morphisms in a category to induced morphisms produced out of a colimit. As a motivating example, consider the following question.

\begin{question} Let $I$ be a small indexing diagram, and let $\alpha,\beta: I \to \Set$ denote two functors, and let $A_i := \alpha(i)$ and $B_i = \beta(i)$ denote the sets at each object $i\in I$. Suppose we have a natural transformation between these functors, consisting of functions $f_i : A_i \to B_i$ for each $i$. \textit{If each $f_i$ is injective, is it true that the induced map $f: \colim \alpha \to \colim \beta$ is injective as well?}
\end{question}

It turns out the answer to this diagram depends on the shape of $I$. It is true if the colimits are \textit{filtered}, which is a condition on the indexing diagram which tells us that it interacts well with finite limits. This motivates the more broad question of what properties of morphisms are preserved under limits and colimits. In general this is a hard question, but it will be important when we investigate fibrations and cofibrations in the category of topological spaces.


\subsection{Stability and closure definitions}



Let $\mathscr{C}$ be a category, not necessarily assumed to be locally small. We start with an easy definition.

\begin{definition}\label{def:closed-under-pullback} Let \textbf{P} be a property of morphisms in $\mathscr{C}$. We say that \textbf{P} is \textit{closed under composition} if, anytime we have two composable morphisms $f: x \to y$ and $g: y \to z$, if both $f$ and $g$ have property \textbf{P}, then the composite $g\circ f$ does as well.
\end{definition}


\begin{definition}\label{def:2-out-of-3} Let \textbf{P} be a property of morphisms in $\mathscr{C}$. We say that \textbf{P} \textit{satisfies 2-out-of-3} if for every commutative diagram of the form
\[ \begin{tikzcd}
    A\rar["f" above]\ar[dr,"g\circ f" below left] & B\dar["g" right]\\
     & C,\\
\end{tikzcd} \]
if any two of $f$, $g$, or $g\circ f$ have property \textbf{P}, then the third does as well.
\end{definition}


\begin{exercise} $\ $
\begin{enumerate}
    \item Prove that isomorphisms satisfy 2-out-of-3.
    \item In the category $\Set$, prove that injections and surjections do not satisfy 2-out-of-3.
\end{enumerate}
\end{exercise}

\begin{definition}\label{def:cancellative-properties} Let \textbf{P} be a property of morphisms.
\begin{enumerate}
    \item We say that \textbf{P} is a \textit{left cancellative property} if any time $g\circ f$ has property \textbf{P}, this implies that $f$ has property \textbf{P} as well.
    \item We say that \textbf{P} is a \textit{right cancellative property} if any time $g\circ f$ has property \textbf{P}, this implies that $g$ has property \textbf{P} as well.
\end{enumerate}
\end{definition}
As a warning to the reader, \textbf{the terminology in} \autoref{def:cancellative-properties} \textbf{is not standard}, although we would advocate for its usage. The notion of left cancellative properties is referred to as (CANC) in \cite[Appendix~C]{GortzWedhorn}, and has many examples in algebraic geometry (e.g. immersions, locally of finite type, purely inseparable, quasi-separated, separated). Many other examples occur under hypotheses on $g$, e.g. if it is separated or unramified. In a broader categorical context, we will see that mono-(resp. epi-)morphisms are left (resp. right) cancellative. 


\begin{definition}\label{def:closed-under-retracts} Let \textbf{P} be a property of morphisms. We say that \textbf{P} is \textit{closed under retracts} if, for any $f: A \to B$ with property \textbf{P}, and any $g: X \to Y$ fitting into a commutative diagram
\[ \begin{tikzcd}
    X\ar[rr,bend left=30,"\id_X" above]\dar["g" left]\rar & A\dar["f"]\rar & X\dar["g" right]\\
    Y\rar\ar[rr,bend right=30,"\id_Y" below] & B\rar & Y,
\end{tikzcd} \]
we have that $g$ has property \textbf{P} as well.
\end{definition}







\begin{definition} Let \textbf{P} be a property of morphisms. Then we say that \textbf{P} is \textit{stable under pullback} (often also called \textit{stable under base change}) if for any pullback diagram of the form
\[ \begin{tikzcd}
    C\rar["j" above]\dar["k" left]\pb & D\dar["f" right]\\
    A\rar["g" below] & B,\\
\end{tikzcd} \]
if $f$ has property \textbf{P}, then $k$ has property \textbf{P} as well. Dually, we say that \textbf{P} is \textit{stable under pushout} if for any pushout diagram of the form
\[ \begin{tikzcd}
    C\rar["j" above]\dar["k" left] & D\dar["f" right]\\
    A\rar["g" below] & B\po,\\
\end{tikzcd} \]
if $k$ has property \textbf{P}, then $f$ has property \textbf{P} as well.
\end{definition}

\begin{exercise}\label{exer:injective-stable-under-pullback} $\ $
\begin{enumerate}
    \item Prove that isomorphisms are always stable under pushout and pullback.
    \item Prove in the category $\Set$ that injective functions are stable under pullback, and surjective functions are stable under pushout.
\end{enumerate}
\end{exercise}


Thus far we have discussed closure properties internal to a category. We might wonder how properties translate across functors.

\begin{definition}\label{def:functor-preserve-reflect-properties} Let $F: \mathscr{C} \to \mathscr{D}$ be a functor, and let \textbf{P} be a property of morphisms.
\begin{enumerate}
    \item We say that $F$ \textit{preserves property} \textbf{P} if any time $f: x \to y$ is a morphisms in $\mathscr{C}$ with property \textbf{P}, we have that $Ff: Fx \to Fy$ has property \textbf{P} as well.

    \item We say that $F$ \textit{reflects property} \textbf{P} if any tie $f:x \to y$ is a morphism in $\mathscr{C}$, if $Ff: Fx \to Fy$ has property \textbf{P}, then $f$ must necessarily have property \textbf{P} as well.
\end{enumerate}
\end{definition}

\begin{remark} We have that $F$ always preserves isomorphisms (this is part of the property of being a functor). It is \textit{not true} that $F$ needs to reflect isomorphisms. For example if $X$ is a topological space with a non-trivial topology, then the map $\id : X \to X_{\text{disc}}$ from $X$ to itself with the discrete topology is not a homeomorphism, however its underlying set map is a bijection. (That is, the forgetful functor $\Top \to \Set$ does not reflect isomorphisms).
\end{remark}

\begin{exercise} If $F$ is full and faithful, it reflects isomorphisms.
\end{exercise}




\subsection{Monomorphisms and epimorphisms}

\autoref{exer:injective-stable-under-pullback} generalizes, as the notions of being injective and surjective in $\Set$ are equivalent to more abstract categorical conditions. We begin by generalizing injections.

\begin{definition}\label{def:monomorphism} Let $\mathscr{C}$ be a category. Then a morphism $f: x \to y$ is a \textit{monomorphism} if any of the following equivalent conditions hold:
\begin{enumerate}
    \item For any pair of morphisms $g,h : z \to x$ so that $f\circ g = f\circ h$, that is:
    \begin{align*}
        z \rightrightarrows x \to y,
    \end{align*}
    we have that $g = h$. (Another way of phrasing this is that $f$ is \textit{left cancellable}).

    \item The following diagram is a pullback:
\[ \begin{tikzcd}
    x\rar["\id_x" above]\dar["\id_x" left]\pb & x\dar["f" right]\\
    x\rar["f" below] & y.
\end{tikzcd} \]

If $\mathscr{C}$ is locally small, there is another equivalent condition:
\end{enumerate}


\begin{enumerate}
\setcounter{enumi}{2}
    \item The induced functor $\Hom_\mathscr{C}(-,x) \xto{f\circ -} \Hom_\mathscr{C}(-,y)$ is a natural injection, meaning that $\Hom_\mathscr{C}(z,x) \xto{f\circ -} \Hom_\mathscr{C}(z,y)$ is an injective function for any $z\in \mathscr{C}$.
\end{enumerate}
\end{definition}

\begin{exercise} Prove that the definitions in \autoref{def:monomorphism} are equivalent.
\end{exercise}

\begin{example} We have that isomorphisms and equalizers are always monomorphisms. Any morphism from a terminal object is a monomorphism.
\end{example}

\begin{exercise} \textit{(Closure properties of monomorphisms)}
\begin{enumerate}
    \item Monomorphisms are closed under composition
    \item Monomorphisms are stable under pullback
    \item Monomorphisms are left cancellative.
\end{enumerate}
\end{exercise}

\begin{exercise}\label{exer:left-adjoints-preserve-monomorphisms} \textit{(Left adjoints preserve monomorphisms)} Let $F: \mathscr{C} \rightleftarrows \mathscr{D} : G$ be an adjunction betweem locally small categories. Then if $f: x\to y$ is a monomorphism in $\mathscr{C}$, we have that $Ff : Fx \to Fy$ is a monomorphism in $\mathscr{D}$. (Hint: use the natural bijection associated to the adjunction, together with \autoref{def:monomorphism}, Definition (iii)).
\end{exercise}

There is a dual notion, called an \textit{epimorphism}.

\begin{definition}\label{def:epimorphism} Let $\mathscr{C}$ be a category. Then a morphism $f: x \to y$ is a \textit{epimorphism} if any of the following equivalent conditions hold:
\begin{enumerate}
    \item For any pair of morphisms $g,h : y \to z$ so that $g\circ f = h\circ f$, that is:
    \begin{align*}
         x \to y \rightrightarrows z,
    \end{align*}
    we have that $g = h$. (Another way of phrasing this is that $f$ is \textit{right cancellable}).

    \item The following diagram is a pushout:
\[ \begin{tikzcd}
    x\rar["f" above]\dar["f" left] & y\dar["\id_y" right]\\
    y\rar["\id_y" below] & y\po.
\end{tikzcd} \]

If $\mathscr{C}$ is locally small, there is another equivalent condition:
\end{enumerate}


\begin{enumerate}
\setcounter{enumi}{2}
    \item The induced functor $\Hom_\mathscr{C}(y,-) \xto{-\circ f} \Hom_\mathscr{C}(x,-)$ is a natural injection, meaning that $\Hom_\mathscr{C}(y,z) \xto{-\circ f} \Hom_\mathscr{C}(x,z)$ is an injective function for any $z\in \mathscr{C}$.
\end{enumerate}
\end{definition}

Epimorphisms satisfy all the dual properties to monomorphisms.

\begin{exercise}\label{exer:properties-of-epimorphisms} \textit{(Properties of epimorphisms)}
\begin{enumerate}
    \item Every isomorphism is an epimorphism, as is every coequalizer.
    \item Any morphism to an initial object is an epimorphism.
    \item Epimorphisms are closed under composition.
    \item Epimorphisms are stable under pushout.
    \item Epimorphisms are right cancellative. 
    \item Right adjoints preserve epimorphisms.
\end{enumerate}
\end{exercise}

As we have hinted at, monomorphisms and epimorphisms generalize the notions of injectivity and surjectivity in $\Set$. We can ask then whether categories which we think of as ``sets with extra data'' have the property that their monomorphisms are just those underlain by injections. Phrased differently, does the forgetful functor $U: \mathscr{C} \to \Set$ reflect mono and epimorphisms? This turns out to be true in some generality, but we must first make rigorous what it means to be a ``set with extra structure.''

\begin{definition}\label{def:concrete-category} A \textit{concrete category} is a locally small category $\mathscr{C}$, together with faithful functor to sets $U : \mathscr{C} \to \Set$. We think of this functor as ``forgetting'' the data.
\end{definition}

\begin{examples}\label{exs:concrete-categories} The following categories are concrete: $\Grp$, $\Poset$, $\Vect$, $\Top$, \todo{add more}.
\end{examples}


\begin{proposition} Any faithful functor reflects monomorphisms and epimorphisms.
\end{proposition}

Thus we see that all the categories in \autoref{exs:concrete-categories} have the property that their monomorphisms are underlying injections, and epimorphisms are underlying surjections. We summarize this in the following table.

\begin{table}[h]
    \centering
    \caption{Examples of monomorphisms and epimorphisms}
    \begin{tabular}{ p{2cm} l  p{5cm}  p{8cm} }
        \toprule
\textbf{Category}      
& \textbf{Monomorphisms}   
& \textbf{Epimorphisms} \\\midrule
$\Set$ & injections & surjections \\\hline

$\Grp$ & underlying injections & underlying surjections  \\\hline	 \bottomrule
    \end{tabular}
\end{table}


\begin{counterexample} We have that the inclusion $\Z \hookto \mathbb{Q}$ is an epimorphism in the category $\Ring$ of unital rings. This tells us that the forgetful functor $U: \Ring \to \Set$ is \textit{not} faithful.
\end{counterexample}

\subsection{Properties of morphisms can induce properties of objects}

Suppose we have a commutative diagram of the form
\[ \begin{tikzcd}
    A\rar["i" above]\ar[dr,"\id_A" below left] & B\dar["r" right]\\
     & A.
\end{tikzcd} \]
In this case we say the object $A$ is a \textit{retract} of $B$. We can ask about what properties of objects descend to their retracts, motivating the following definition.

\begin{definition}\label{def:property-objects-closed-under-retracts} Let \textbf{O} be a property of objects in $\mathscr{C}$ (we can just think of this as some subclass of the class of objects). Then we say that \textbf{O} is \textit{closed under retracts} if, any time $B \in \mathbf{O}$, and $A$ is a retract of $B$, we have that $A\in \mathbf{O}$ as well.
\end{definition}

We can ask about how a property of objects could potentially be related to a property of morphisms. In particular, we could ask about how properties of morphisms might induce properties of objects. Our perspective on this is stolen from algebraic geometry, in which a scheme is defined to have a property if and only if its structure morphism has the associated property. In the world of schemes and varieties, there basically don't exist properties for objects, only properties for morphisms. This is a feature of life with a terminal object. In many situations, such as topological spaces, we have the same luxury.

\begin{terminology}\label{term:inducing-properties-of-objects-from-properties-of-morphisms} Let $\mathscr{C}$ be a category with a terminal object $\ast$, and let \textbf{P} be a property of morphisms in $\mathscr{C}$. Then we have an \textit{induced property of objects} \textbf{O}, defined by saying that $X\in \mathbf{O}$ if the unique morphism $X \xto{!} \ast$ lies in \textbf{P}.
\end{terminology}

\begin{exercise} Let $\mathscr{C}$ be a category with a terminal object. Prove that if \textbf{P} is a property of morphisms closed under retracts in the sense of \autoref{def:closed-under-retracts}, then the induced property of objects is closed under retracts in the sense of \autoref{def:property-objects-closed-under-retracts}.
\end{exercise}

\begin{remark} The passage from fibrations to fibrant objects follows \autoref{term:inducing-properties-of-objects-from-properties-of-morphisms}.
\end{remark}






\subsection{Preservation properties under limits and colimits}

\begin{definition}\label{def:naturally-P} Let $I$ be an indexing category, let $\mathscr{C}$ be a category, and let \textbf{P} be a property of morphisms in $\mathscr{C}$. Let $f_1, f_2 : I \to \mathscr{C}$ be two diagrams in $\mathscr{C}$, and let $\eta: f_1 \Rightarrow f_2$ be a natural transformation between them. We say that $\eta$ \textit{has property} \textbf{P} \textit{levelwise} (or \textit{has property} \textbf{P} \textit{naturally}) if each of the components $\eta_c : f_1(c) \to f_2(c)$ has property \textbf{P}. As some examples:
\begin{enumerate}
    \item a \textit{natural isomorphism} is a natural transformation whose components are isomorphisms.
    \item a \textit{natural monomorphism} is a natural transformation whose components are monomorphisms.
\end{enumerate}
\end{definition}



\begin{definition}\label{def:colimit-preserve-property} Let $I$ be an indexing category, and let \textbf{P} be a property of morphisms in $\mathscr{C}$. We say that \textbf{P} \textit{is preserved under} $I$\textit{-shaped colimits} if, any time we have a natural transformation $\eta: f_1 \Rightarrow f_2$ which has property \textbf{P} levelwise, we have that the induced map
\begin{align*}
    \colim_I(f_1) \to \colim_I(f_2)
\end{align*}
is in \textbf{P} as well (assuming both colimits exist). Dually we have a notion of preservation under $I$-shaped limits.
\end{definition}

\begin{example} We have that isomorphisms are preserved under arbitrary limits and colimits.
\end{example}

\begin{example} If $I = \bullet\ \bullet$ is a discrete category on two points, we say that a property \textbf{P} is \textit{preserved under products} (resp. \textit{coproducts}) if it is preserved under $I$-shaped limits (resp. colimits).
\end{example}

\begin{proposition}\label{prop:monos-preserved-under-products} We have that
\begin{enumerate}
    \item Monomorphisms are preserved under products
    \item Epimorphisms are preserved under coproducts.
\end{enumerate}
\end{proposition}
\begin{proof} We will prove (1) and remark that (2) follows formally. Let $f: A \to B$ and $g: C \to D$ be monomorphisms. Then assuming their products exist, there is an induced map $(f \times g) : A \times C \to B \times D$. We claim that this is a monomorphism as well. Via \autoref{def:monomorphism}, we can restate the statement that $f$ and $g$ are monomorphisms into the statement that these diagrams are pullbacks
\[ \begin{tikzcd}
    A\rar\dar\pb & A\dar\\
    A\rar & B
\end{tikzcd} \quad\quad  \begin{tikzcd}
    C\rar\dar\pb & C\dar\\
    C\rar & D.
\end{tikzcd} \]
Since limits commute, we have that taking a product of the two pullbacks is the same as the pullback of the product of the two diagrams, that is, the following diagram is a pullback
\[ \begin{tikzcd}
    A \times C\rar["\id \times \id"]\dar["\id \times \id" left]\pb & A \times C\dar["f \times g" right]\\
    A \times C \rar["f \times g" below] & B \times D.
\end{tikzcd} \]
Thus $f \times g$ is a monomorphism.
\end{proof}


This is an instance of a more general phenomenon, namely functors preserving limits and colimits.

\begin{definition}\label{def:preserve-reflect-create-limits} Let $F:\mathscr{C} \to \mathscr{D}$ be a functor, and let $I$ be an indexing diagram.
\begin{enumerate}
    \item We say that $F$ \textit{preserves} $I$-shaped limits if, for any functor $j: I \to \mathscr{C}$, we have that the natural map is an isomorphism
    \begin{align*}
        F(\lim(j)) \cong \lim(F\circ j).
    \end{align*}
    That is, a limit cone over $j$ is sent to a limit cone over $F\circ j$.

    \item We say that $F$ \textit{reflects} $I$-shaped limits if, for any functor $j: I \to \mathscr{C}$, if we have that $F(c)$ is the limit of $F\circ j$, then $c$ is a limit of $j$. Phrased differently, a limit cone over $F\circ j$ in the image of $F$ must have come from a limit cone over $j$.

    \item We say $F$ \textit{creates} limits if it preserves and reflects them.
\end{enumerate}
We have analogous definitions for colimits.
\end{definition}

This relates to some things we have already seen: e.g. \autoref{prop:LAPC} said that left adjoints preserve colimits, and right adjoints preserve limits.

\begin{proposition}\label{prop:hom-functors-preserve-limits} Hom functors preserve limits in both arguments. That is, if $\mathscr{C}$ is a locally small category, then the bifunctor
\begin{align*}
    \Hom : \mathscr{C}^\op \times \mathscr{C} &\to \Set
\end{align*}
preserves limits in both arguments (limits in $\mathscr{C}^\op$ are colimits in $\mathscr{C}$).
\end{proposition}


\begin{corollary}\label{cor:Yoneda-embedding-preserves-limits} Let $\mathscr{C}$ be locally small. Then the Yoneda embedding $y: \mathscr{C} \hookto \Fun(\mathscr{C}^\op, \Set)$ preserves and reflects all limits.
\end{corollary}

\begin{proposition}\label{prop:fully-faithful-functor-reflects-limits-colimits} A fully faithful functor reflects all limits and colimits.
\end{proposition}


\begin{remark}\label{rmk:labelname} If $F$ is a functor preserving all limits over the diagram $I = \bullet \to \bullet \from \bullet$, we say that it \textit{preserves pullbacks}. Dually if $F$ preserves limits over the span diagram $J = \bullet \from \bullet \to \bullet$, we say it is \textit{preserves pushouts}.
\end{remark}

\begin{exercise}\label{exer:pullback-preserving-functor-preserves-monos} Let $F: \mathscr{C}\to \mathscr{D}$ be a functor.
\begin{enumerate}
    \item If $F$ preserves pullbacks, then it preserves monomorphisms.
    \item If $F$ preserves pushouts, then it preserves epimorphisms.
\end{enumerate}
\end{exercise}

Thus far we have defined two notions of preservation of properties. Namely, properties preserved under functors (as in \autoref{def:functor-preserve-reflect-properties}), and properties preserved under limits and colimits (as in \autoref{def:colimit-preserve-property}). We can see that these are both instances of the same idea.

\subsection{Colimits and limits as functors}

In order to relate these two notions of preservation, we should relate one to the other. In particular we claim that the most general notion of preservation of properties of morphisms is under a functor. In order to see that this encapsulates the other definition, we should view the construction of limits and colimits as a functor in some sense.

Let $\mathscr{C}$ be a category, and $I$ be an indexing category which we want to take limits and colimits over. Then there is a \textit{diagonal functor}
\begin{align*}
    \Delta : \mathscr{C} &\to \Fun(I, \mathscr{C}).
\end{align*}
This sends an object $c$ to the functor $\Delta_c: I \to \mathscr{C}$, where $\Delta_c(i) = c$ for all $i\in I$, and $\Delta_c(i \to i') = \id_c$. That is, $\Delta_c$ is really a constant functor at the object $c\in \mathscr{C}$. For a morphism $f:c\to c'$, we have an induced natural transformation  $\Delta_c \Rightarrow \Delta_{c'}$, all of whose components are $f$. 

\begin{center}
    [[todo]]    
\end{center}

\subsection{Filtered limits and colimits}

\begin{center}
    [[todo]]    
\end{center}




\section{Unsorted subsections on category theory}

\subsection{Conventions}

We record some standard notation for use in these notes.

\begin{notation}\label{nota:initial-terminal-objects} We denote an \textit{initial object} in a category $\mathscr{C}$ by $\emptyset$. We denote a \textit{terminal object} in a category $\mathscr{C}$ by $\ast$. If $X \in \mathscr{C}$ is arbitrary, it receives a unique map from the initial object and to the terminal object. We decorate each of these with a shriek:
\begin{align*}
    \emptyset &\xto{!} X \\
    X &\xto{!} \ast.
\end{align*}
\end{notation}

\begin{notation}\label{nota:diagonal-map} Let $\mathscr{C}$ be a category, and let $X\in \mathscr{C}$ be an arbitrary object. If the product $X \times X$ exists, there is a \textit{diagonal map}, which we denote by
\begin{align*}
    X \xto{\Delta} X \times X,
\end{align*}
defined to be the unique map provided to us by the universal property of the product
\[ \begin{tikzcd}
     & X\dar\ar[ddl,bend right=10,"\id_X" left]\ar[ddr,bend left=10,"\id_X" right]\dar[dashed,"\Delta" right] & \\
     & X \times X\ar[dl,"\pr_1" below right]\ar[dr,"\pr_2" below left] & \\
    X &  & X.
\end{tikzcd} \]
\end{notation}
\begin{notation}\label{nota:fold-map} Let $\mathscr{C}$ be a category, and let $X\in \mathscr{C}$ be an arbitrary object. If the coproduct $X \amalg X$, there is a \textit{fold map}, which we denote by
\begin{align*}
    X \amalg X \xto{\nabla} X,
\end{align*}
defined to be the unique map provided to us by the universal property of the coproduct
\[ \begin{tikzcd}
    X\ar[dr,"i_1" above right]\ar[ddr,bend right=10,"\id_X" left] &  & X\ar[dl,"i_2" above left]\ar[ddl,bend left=10,"\id_X" right]\\
     & X \amalg X\dar[dashed,"\nabla" right]& \\
     & X. & \\
\end{tikzcd} \]
\end{notation}

The terminology for the \textit{diagonal} map comes from the world of sets (or concrete categories more generally), where the function $\Delta: X \to X \times X$ is defined elementwise by $x \mapsto (x,x)$. That is, its image is the diagonal in the product. The \textit{fold} map comes from the mental image of taking two disjoint copies of $X$ and folding them over one another into one copy of $X$.

\begin{notation}\label{nota:product-coproduct-of-morphisms} Let $\mathscr{C}$ be a category with products, and let $f: A \to B$ and $g: C \to D$ be two morphisms. Then we denote by
\begin{align*}
    f \times g : A \times C \to B \times D
\end{align*}
the morphism given via the universal property
\[ \begin{tikzcd}
     & A \times B\ar[dl]\ar[dr]\dar[dashed,"f \times g"] & \\
    A\dar["f" left] & C \times D\ar[dl]\ar[dr] & B\dar["g" right]\\
    C &  & D.
\end{tikzcd} \]
Similarly if $\mathscr{C}$ is a category with coproducts, we denote by
\begin{align*}
    f \amalg g : A \amalg C \to B \amalg D
\end{align*}
the morphism provided by the universal property
\[ \begin{tikzcd}
    A\ar[dr]\dar["f" left] &  & B\ar[dl]\dar["g" right]\\
    C\ar[dr] & A\amalg B\dar[dashed,"f \amalg g"] & D\ar[dl]\\
     & C \amalg D. &
\end{tikzcd} \]

\end{notation}




\subsection{Laws for pullbacks and pushouts}

Frequently we will be asked to glue together commutative squares and talk about whether the composite square is a pullback and/or whether the individual squares are pullbacks. This is discussed in the so-called \textit{pasting law}.

\begin{proposition}\label{prop:pasting-law-pullbacks-pushouts} \textit{(Pasting law for pullbacks and pushouts)} Let $\mathscr{C}$ be an arbitrary category, and consider a commutative diagram of the following form
\[ \begin{tikzcd}
    \bullet\rar\dar & \bullet\rar\dar & \bullet\dar\\
    \bullet\rar & \bullet\rar & \bullet.
\end{tikzcd} \]
\begin{enumerate}
    \item If the right square is a pullback, then the total square is a pullback if and only if the left square is a pullback.
    \item If the left square is a pushout, then the total square is a pushout if and only if the right square is a pushout.
\end{enumerate}
\end{proposition}

We also have the ``magic square,'' which is generally stated for varieties or schemes, but holds in a more broad context.

\begin{proposition}\label{prop:magic-pullback-square} \textit{(Magic pullback square)} Let $\mathscr{C}$ be a locally small category with products and pullbacks. Let $f: X \to Z$, and $g: Y \to Z$ be morphisms in $\mathscr{C}$. Then the following square is a pullback
\[ \begin{tikzcd}
    X \times_Y Z\rar\dar\pb & X \times Y\dar["f \times g" right]\\
    Z\rar["\Delta" below] & Z \times Z.
\end{tikzcd} \]
\end{proposition}
\begin{proof} We first check that this holds if $\mathscr{C} = \Set$. This is certainly true, since elements of $(X \times Y)\times_{Z \times Z} Z$ are precisely elements $(x,y) \in X \times Y$ so that $f(x) = g(y)$. Then for any locally small category $\mathscr{C}$, we have that this diagram is a levelwise pullback in $\Fun \left( \mathscr{C}^\op, \Set \right)$. Finally we remark that under the Yoneda embedding $y : \mathscr{C} \hookto \Fun(\mathscr{C}^\op, \Set)$, limits are reflected.
\end{proof}

The magic pullback square is generally stated in the following form.

\begin{corollary}\label{cor:magic-square-over-S} Let $\mathscr{C}$ be a locally small category, and $S\in \mathscr{C}$ be any element. Then if $f: X \to Z$ and $g : Y \to Z$ are morphisms over $S$, we have a pullback square
\[ \begin{tikzcd}
    X \times_Y Z\rar\dar\pb & X \times_S Y\dar["f \times g"]\\
    Z\rar["\Delta" below] & Z \times_S Z.
\end{tikzcd} \]
\end{corollary}



